\section{Definitions and Properties of Symplectic Groups} \label{gorilla}

In this section, we first detail the basic definitions and properties of symplectic groups, and we then proceed to prove a few group-theoretic lemmas that are used in our proof of the main result of this paper, Theorem~\ref{theorem:main}. The reader should feel free to proceed to Section~\ref{section:background} upon reading the statements of Propositions~\ref{theorem:adelic-surjective-subset} and~\ref{theorem:commutator-open}.

\subsection{Symplectic Groups}\label{subsection:stimpy}

Fix a commutative ring $R$, a free $R$-module $M$ of rank $2g$ for some positive integer $g$, and a non-degenerate alternating bilinear form $\langle -, - \rangle \colon M \times M \to R$. Define the {\it general symplectic group} (alternatively, the \emph{group of symplectic similitudes}) $\GSp(M)$ to be the subgroup of $\on{GL}(M)$ consisting of all $R$-automorphisms $S$ such that there exists some $m_S \in R^\times$, called the {\it multiplier} of $S$, satisfying $\langle S v, Sw \rangle = m_S \cdot \langle v, w \rangle$ for all $v, w \in M$. One readily observes that the {\it mult} map
\begin{align*}
	\mult \colon \GSp(M) & \rightarrow R^\times \\
	S & \mapsto m_S
\end{align*}
is a group homomorphism, and its kernel is the {\it symplectic group}, denoted by $\Sp(M)$.

By choosing a suitable $R$-basis for $M$, we can arrange for the corresponding matrix of the inner product $\langle -, - \rangle$ to be given by
$$\Omega_{2g} = \left[\begin{array}{c|c} 0 & \id_g \\ \hline -\id_g & 0\end{array}\right],$$
where $\id_g$ denotes the $g \times g$ identity matrix. From this choice of basis we obtain an identification $\GL(M) \simeq \GL_{2g}(R)$. We then define $\GSp_{2g}(R)$ to be the image of $\GSp(M)$ and $\Sp_{2g}(R)$ to be the image of $\Sp(M)$ under this identification. Let $\det \colon \GL_{2g}(R) \to R^\times$ be the determinant map. Since the diagram
\begin{center}
\begin{tikzcd}
\GSp(M) \arrow{r}{\sim} \arrow[swap]{rd}{\on{mult}^g} &  \GSp_{2g}(R) \arrow{d}{\on{det}} \\
& R^\times
\end{tikzcd}
\end{center}

\noindent commutes, where the diagonal map is the multiplier map raised to the $g^{\mathrm{th}}$ power, one deduces that $\GSp_{2g}(R)$ is in fact the subgroup of $\GL_{2g}(R)$ consisting of all invertible matrices $S$ satisfying $S^T \Omega_{2g} S = (\on{mult} S) \, \Omega_{2g}$ and that $\Sp_{2g}(R) = \ker(\on{mult} \colon \GSp_{2g}(R) \to R^\times)$.

Let $\on{Mat}_{2g \times 2g}(R)$ denote the space of $2g \times 2g$ matrices with entries in $R$. In subsequent subsections, we will make heavy use of the ``Lie algebra'' $\mf{sp}_{2g}(R)$, which is defined by
		\begin{align*}
			\mf{sp}_{2g}(R) &\defeq \{M \in \on{Mat}_{2g \times 2g}(R) : M^T \Omega_{2g} + \Omega_{2g} M = 0 \}.
		\end{align*}
		It is easy to see that $M^T \Omega_{2g} + \Omega_{2g}M = 0$ is equivalent to $M$ being a block matrix with $g \times g$ blocks of the form
		\[
			M = \left[\begin{array}{c|c} A & B \\ \hline C & -A^T \end{array}\right],
		\]
		where $B$ and $C$ are symmetric.

For the purpose of studying Galois representations associated to PPAVs, we will be primarily interested in the cases where the ring $R$ is the profinite completion $\wh{\ZZ}$ of $\ZZ$, the ring of $\ell$-adic integers $\ZZ_{\ell}$ for a prime number $\ell$, or the finite cyclic ring $\ZZ / m \ZZ$ for a positive integer $m$.
Note in particular that we have the identifications
\begin{equation}\label{orientation1}
\GSp_{2g}(\ZZ_\ell)   \simeq  \varprojlim_k \GSp_{2g}(\ZZ/ \ell^k \ZZ) \quad \text{and}
\end{equation}
\begin{equation}\label{orientation2}
\prod_{\text{prime } \ell} \GSp_{2g}(\ZZ_\ell) \simeq  \GSp_{2g}(\wh{\ZZ}) \simeq  \varprojlim_m \GSp_{2g}(\ZZ / m \ZZ).
\end{equation}
From~\eqref{orientation1} and~\eqref{orientation2}, we obtain the $\ell$-adic projection map $\pi_\ell \colon \GSp_{2g}(\wh{\ZZ}) \twoheadrightarrow \GSp_{2g}(\ZZ_\ell)$ and the mod-$m$ reduction map $r_m \colon \GSp_{2g}(\wh{\ZZ}) \twoheadrightarrow \GSp_{2g}(\ZZ / m \ZZ)$. Observe that~\eqref{orientation1} and~\eqref{orientation2} both hold with $\GSp_{2g}$ replaced by $\Sp_{2g}$.



\subsection{Notation}
\label{subsection:notation}

In what follows, we study subquotients of $\Sp_{2g}(\wh{\ZZ})$, $\Sp_{2g}(\ZZ_\ell)$, and $\Sp_{2g}(\ZZ/ \ell^k \ZZ)$ for $\ell$ a prime number and $k$ a positive integer. We use the following notational conventions:
\begin{itemize}
\item Let $H\subset \Sp_{2g}(\wh{\ZZ})$ be a closed subgroup.
\item Let $H_{\ell} \defeq \pi_\ell(H) \subset \Sp_{2g}(\ZZ_\ell)$ be the $\ell$-adic reduction of $H$. More generally, for any set $S$ of prime numbers, let $H_S$ denote the projection of $H$ onto $\prod_{\ell \in S} \Sp_{2g}(\wh{\ZZ})$.
\item Let $H(m) = r_{m}(H) \subset \Sp_{2g}(\ZZ/m \ZZ)$ be the mod-$m$ reduction of $H$. We often take $m = \ell^k$.
\item Let $\Gamma_{\ell^k} = \ker(\Sp_{2g}(\ZZ_\ell) \to \Sp_{2g}(\ZZ/ \ell^k \ZZ))$. Notice that the map $M \mapsto \id_{2g} + \ell^k M$ gives an isomorphism of groups
$$\mf{sp}_{2g}(\ZZ/ \ell \ZZ) \simeq \ker(\Sp_{2g}(\ZZ/\ell^{k+1} \ZZ) \to \Sp_{2g}(\ZZ/ \ell^k \ZZ))$$
for every $k \geq 1$, so we will use $\mf{sp}_{2g}(\mathbb Z/\ell\ZZ)$ to denote the above kernel.
\item For any group $G$, let $[G,G]$ be its commutator subgroup, and let $G^{\ab} \defeq G/{ {[G,G]}}$ be its abelianization.
\item For any group $G$, let $\quo(G)$ the set of isomorphism classes of finite non-abelian simple quotients of $G$, and let $\occ(G)$ be the set of isomorphism classes of finite non-abelian simple \emph{sub}quotients of $G$.
\item For any positive integer $m$, let $S_m$ denote the symmetric group on $m$ letters.
\end{itemize}

\subsection{Generalized Goursat's Lemma}

In Sections~\ref{subsection:closed-subgroups} and~\ref{subsection:open-subgroups}, it will be crucial for us to have a theorem that allows us to express a subgroup of $\Sp_{2g}(\wh{\ZZ})$ as (roughly) the product of its $\ell$-adic reductions. A natural tool for doing this is Goursat's lemma, but in much of the literature (e.g.,~\cite[Lemma 5.2.2]{ribbit} and~\cite[Lemma A.4]{zywina2010elliptic}), this result is only stated for \emph{finite} products of \emph{finite} groups. This section is devoted to proving Theorem~\ref{theorem:goursat}, which generalizes Goursat's lemma to apply in the setting that we need, namely for \emph{countable} products of \emph{profinite} groups.
	
	\begin{lemma} \label{lemma:product-quotient}
		Let $G = \prod_{i=1}^n G_i$ be a product of profinite groups. Then every finite simple quotient of $G$ is a finite simple quotient of $G_i$ for some $i$, and vice versa.
	\end{lemma}
	\begin{proof}
		Consider a finite simple quotient $\phi: G \twoheadrightarrow H$. Since each $G_i \subset G$ is normal, the image $\phi(G_i) \subset H$ is also normal. For any $i$, if $\phi(G_i)$ is larger than $\{1\}$, then it equals $H$ since $H$ is simple, and the composition
		\(
			G_i \hookrightarrow G \twoheadrightarrow H
		\)
		expresses $H$ as a quotient of $G_i$. If no such $i$ exists, then $\ker \phi = G$, contradiction. The ``vice versa'' statement is obvious.
	\end{proof}
	
	\begin{theorem} \label{theorem:goursat}
		Let $A$ be a countable set, and suppose $\{G_\alpha\}_{\alpha \in A}$ is a collection of profinite groups such that, for all pairs $\alpha, \beta \in A$ with $\alpha \neq \beta$, the groups $G_\alpha$ and $G_\beta$ have no finite simple quotients in common. Let $G := \prod_{\alpha \in A} G_\alpha$, and let $\pi_\alpha : G \to G_\alpha$ be the natural projections. If $H \subset G$ is a closed subgroup with $\pi_\alpha(H) = G_\alpha$ for all $\alpha \in A$, then $H = G$.
	\end{theorem}
	\begin{proof}
		First take $A = \{1, 2\}$, so that $G = G_1 \times G_2$. The subgroup
		\(
			N_1 \times \{1\} \defeq (G_1 \times \{1\}) \cap H \subset G
		\)
		is normal because $\pi_1(H) = G_1$. This means $N_1$ is a normal subgroup of $G_1$. Similarly for the subgroup $\{1\} \times N_2$. With these definitions, the closed subgroup
		\(
			H / (N_1 \times N_2) \subset (G_1 / N_1) \times (G_2 / N_2)
		\)
		surjects onto each factor via the natural projections. We have thereby reduced to the case $N_1 = N_2 = 0$. By \cite[Lemma 5.2.1]{ribbit}, we know that $G_1 \simeq G_2$ as profinite groups. The result follows because two isomorphic profinite groups have a nontrivial finite simple quotient in common (and any quotient of $G_i / N_i$ is \emph{a priori} a quotient of $G_i$).
		
		Now take $A = \{1, 2, \ldots, n\}$ for $n \ge 3$, and suppose (by induction) that the result has been proven for $n-1$. For any $H \subset G = \prod_{i=1}^n G_i$ satisfying the hypotheses of the theorem, let $H'$ be the image of $H$ under the projection $G \twoheadrightarrow \prod_{i=1}^{n-1} G_i$. Then $H'$ satisfies the hypotheses for $n-1$, so we conclude that $H' = \prod_{i=1}^{n-1} G_i$. By Lemma~\ref{lemma:product-quotient}, the groups $\prod_{i=1}^{n-1} G_i$ and $G_n$ have no finite simple quotients in common, so the $n = 2$ case tells us that $H = G$.
		
		The only remaining case is $A = \{1, 2, \ldots\}$. Consider $H \subset G$ satisfying the hypotheses of the theorem. For each $n$, let $H(n)$ be the image of $H$ under the projection $G \twoheadrightarrow \prod_{i=1}^{n} G_i$. By the finite case prove above, we know that $H(n) = \prod_{i=1}^n G_i$ for each $n \ge 1$. Fix an element $g \defeq (g_i)_{i \ge 1} \subset G$, and define a sequence $\{h_1, h_2, \ldots\}$ of elements of $H$ as follows: let $h_n$ be any element of $H$ whose image in $\prod_{i=1}^n G_i$ equals $(g_1, \ldots, g_n)$. In the product topology, $\lim_{n \rightarrow \infty} h_n = g$, so $g \in H$ since $H$ is closed. Since $g \in G$ was arbitrary, \mbox{we conclude that $H = G$.}
	\end{proof}

\subsection{Closed Subgroups of $\Sp_{2g}(\wh{\ZZ})$}
\label{subsection:closed-subgroups}
	
	As before, let $H \subset \Sp_{2g}(\zh)$ be a closed subgroup. Our main result of this section is Proposition~\ref{theorem:adelic-surjective-subset}, which shows that properties of $H$ can be deduced from corresponding properties of the $\ell$-adic reductions $H_\ell \subset \Sp_{2g}(\bz_\ell)$ as $\ell$ ranges over the prime numbers.
	We use Proposition~\ref{theorem:adelic-surjective-subset} crucially in our proof of the main theorem, Theorem~\ref{theorem:main},
	and more specifically in the proof of Proposition~\ref{lemma:ab-cyc}.

	Our strategy is to combine Goursat's lemma with the observation that the groups $\Sp_{2g}(\bz_\ell)$ have distinct sets of possible simple quotients as $\ell$ varies. We shall make use of the following version of Goursat's Lemma, which we apply in the proof of Proposition~\ref{theorem:adelic-surjective-subset} to determine a subgroup of $\Sp_{2g}(\widehat{\mathbb Z})$ from its $\ell$-adic images.

	The next lemma enables us to verify the conditions required for applying Goursat's
	Lemma:\vspace*{-0.2in}
	\begin{lemma}
		\label{lemma:simple-quotients-of-symplectic-group}
       If $g > 2$ or $\ell > 2$, we have $\quo(\Sp_{2g}(\ZZ_\ell)) = \{\on{PSp}_{2g}(\ZZ/ \ell \ZZ)\}$. Moreover, for all $g \geq 2$, we have $\quo(\Sp_{2g}(\mathbb Z_\ell)) \cap \quo(\Sp_{2g}(\mathbb Z_{\ell'})) = \varnothing$ if $\ell \neq \ell'$.
	\end{lemma}
	\begin{proof}
		Since $\Gamma_\ell$ is a pro-$\ell$ group, we have that
		$\quo (\Sp_{2g}(\mathbb Z_\ell)) = \quo(\Sp_{2g}(\mathbb Z/ \ell \ZZ))$.
		Furthermore, quotienting by $\{\pm \id_{2g}\}$, we have that
		$\quo(\Sp_{2g}(\mathbb Z/ \ell \ZZ)) = \quo(\Sp_{2g}(\mathbb Z/ \ell \ZZ)/\left\{ \pm \id_{2g} \right\})$. By~\cite[Theorem 3.4.1]{omeara1978symplectic}, we have $\Sp_{2g}(\mathbb Z/\ell \ZZ)/\left\{ \pm \id_{2g} \right\} = \PSp_{2g}(\ZZ/ \ell \ZZ)$ is simple for $g > 2$ or $\ell > 2$. It follows that $\quo(\Sp_{2g}(\ZZ_\ell)) = \{\on{PSp}_{2g}(\ZZ/ \ell \ZZ)\}$ in this case.

To finish the proof, note that $\quo(\Sp_{2g}(\mathbb Z_\ell)) \cap \quo(\Sp_{2g}(\mathbb Z_{\ell'})) = \varnothing$ for $g > 2$ or $\ell, \ell' > 2$ because $\PSp_{2g}(\bz / \ell \bz) \neq \PSp_{2g}(\bz / \ell' \bz)$ for $\ell \neq \ell'$ because their orders are different. The only remaining case is where $g = 2$, $\ell = 2$, and $\ell' > 2$. In this case, observe that $\PSp_{2g}(\bz / \ell' \ZZ) \notin \quo( \Sp_{2g}(\bz / 2 \ZZ))$ for $\ell' > 2$, since the order of $\PSp_{2g}(\bz / \ell \ZZ)$ exceeds that of $\Sp_{2g}(\bz / 2 \ZZ)$.
	\end{proof}
	
We are now ready to prove Proposition~\ref{theorem:adelic-surjective-subset},
assuming Proposition~\ref{theorem:truncate}.
\begin{proposition}
		\label{theorem:adelic-surjective-subset}
		Let $G \subset \Sp_{2g}(\zh)$ be an open subgroup. There exists a positive integer $M$ such that, for every closed subgroup $H \subset G$, we have $H = G$ if and only if $H(M) = G(M)$ and $H(\ell) = \Sp_{2g}(\bz / \ell \ZZ)$ for every prime $\ell \nmid M$.
	\end{proposition}
\subsubsection*{Idea of Proof}
The idea of the proof is to find a sufficiently large $M$
so that if $H(M) = G(M)$ then $H_{\{\ell \hspace{.05cm}\nmid \hspace{.05cm}M\}} = G_{\{\ell \hspace{.05cm} \nmid\hspace{.05cm} M\}}$,
which will reduce the problem to proving
Proposition~\ref{theorem:truncate}.

	\begin{proof}[Proof assuming Proposition~\ref{theorem:truncate}]
		Again, the case where $g = 1$ is handled in~\cite[Lemma 7.6]{zywina2010hilbert}, so take $g \geq 2$. Let $p$ be the largest prime such that $G(p) \neq \Sp_{2g}(\bz / p \ZZ)$.
		Observe that the groups $\Gamma_{\ell^k}$ are open in $\Sp_{2g}(\ZZ_\ell)$ because they have finite index in $\Sp_{2g}(\ZZ_\ell)$. Since $G \subset \Sp_{2g}(\zh)$ is open, the group $G_{\{\ell \le p\}} \subset \prod_{\ell \le p} \Sp_{2g}(\ZZ_\ell)$ is open too, so there exist exponents $e(\ell) \ge 1$ with the property that
		\[
			\prod_{\ell \le p} \Gamma_{\ell^{e(\ell)}} \subset G_{\{\ell \le p \}}.
		\]
		Since the groups $\Gamma_{\ell^k}$ are finitely generated pro-$\ell$ open normal subgroups of $\GSp_{2g}(\ZZ_\ell)$, condition (ii) from~\cite[Proposition 10.6]{serre1989lectures} is satisfied. Hence,
		the equivalence of conditions (ii) and (iv) from~\cite[Proposition 10.6]{serre1989lectures} implies
		that the Frattini subgroup defined by
		\begin{align*}
		\Phi(G_{\{\ell \le p\}}) \defeq \bigcap_{\substack{S \subset G_{\{\ell \le p\}} \\ S \text{ maximal closed in  }G_{\{\ell \le p\}}}} S
		\end{align*}
		is open and normal in $G_{\{\ell \le p\}}$. This means we can find exponents $e'(\ell) \ge 1$ such that
		\[
			\prod_{\ell \le p} \Gamma_{\ell^{e'(\ell)}} \subset \Phi(G_{\{\ell \le p\}}).
		\]
		Define $M \defeq \prod_{\ell \le p} \ell^{e'(\ell)}$. Then $H(M) = G(M)$ implies that $H_{\{\ell \leq p\}} = G_{\{\ell \leq p\}}$.
		
		Now take $H$ satisfying $H(M) = G(M)$ and $H(\ell) = \Sp_{2g}(\bz / \ell \ZZ)$ for every prime $\ell \nmid M$. We have that
		\[
			H \subset G \subset H_{\{\ell \le p\}} \times \prod_{\ell > p} \Sp_{2g}(\bz_\ell).
		\]
		To show that $H = G$, we need only verify
		\[
			H = H_{\{\ell \le p\}} \times \prod_{\ell > p} \Sp_{2g}(\bz_\ell),
		\]
		but this follows immediately from Proposition~\ref{theorem:truncate}.
	\end{proof}
We now complete the proof of Proposition~\ref{theorem:adelic-surjective-subset} by proving
Proposition~\ref{theorem:truncate}.

		\begin{proposition}
		\label{theorem:truncate}
		Let $g \geq 2$ and let $H \subset \Sp_{2g}(\widehat {\mathbb Z})$ be a closed subgroup. Suppose there is a prime number $p \ge 2$ so that $H(\ell) = \Sp_{2g}(\ZZ/ \ell \ZZ)$ for all $\ell > p$. Then we have that
		\begin{equation}\label{meanttofly}
		H = H_{\{\ell \le p\}} \times \prod_{\ell > p} \Sp_{2g}(\ZZ_\ell).
		\end{equation}
	\end{proposition}
\subsubsection*{Idea of Proof}
The idea of the proof is to apply Goursat's Lemma to conclude that if the group surjects onto each factor,
then it surjects onto the product.
We verify the hypotheses of Goursat's Lemma,
using Lemma~\ref{lemma:simple-quotients-of-symplectic-group}, and the fact that all simple quotients of $H_{\{\ell \leq p\}}$
have smaller order than $\PSp_{2g}(\mathbb Z_\ell)$ for $\ell > p$.
	\begin{proof}
		The case where $g = 1$ is handled by~\cite[Lemma 7.6]{zywina2010hilbert}, so take $g \geq 2$. By~\cite[Theorem 1]{landesman-swaminathan-tao-xu:lifting-symplectic-group}, the fact that $H(\ell) = \Sp_{2g}(\ZZ/ \ell \ZZ)$ implies that $H_\ell = \Sp_{2g}(\ZZ_\ell)$ for all $\ell > p$.

  The proposition follows upon applying Theorem~\ref{theorem:goursat} to the product $H_{\{\ell \le p\}} \times \prod_{\ell > p} \Sp_{2g}(\ZZ_\ell)$. However, to apply it, we must check that the sets $\quo(H_{\{\ell \le p\}})$ and $\quo(\Sp_{2g}(\ZZ_\ell))$ for $\ell > p$ are all pairwise disjoint. By Lemma~\ref{lemma:simple-quotients-of-symplectic-group}, it suffices to show that $\quo(H_{\{\ell \le p\}}) \cap \quo(\Sp_{2g}(\ZZ_\ell)) = \varnothing$ for any fixed $\ell > p$. Our strategy for checking this condition is to bound the sizes of the groups appearing in $\quo(H_{\{\ell \le p\}})$. First, observe that
		\[
			\quo(H_{\{\ell \le p\}}) \subset \occ\left( \prod_{\ell \le p} \Sp_{2g}(\bz_\ell)\right) = \bigcup_{\ell \le p} \occ(\Sp_{2g}(\bz_\ell)),
		\]
		where the last step follows from the first displayed equation of~\cite[p.\ IV-25]{serre1989abelian}.
But
\(
			\occ(\Sp_{2g}(\bz_\ell)) = \occ(\Gamma_\ell) \cup \occ(\Sp_{2g}(\bz / \ell \ZZ))
		\),
and $\occ(\Gamma_\ell) = \varnothing$ because $\Gamma_\ell$ is a pro-$\ell$ group, so
		\(
			\occ(\Sp_{2g}(\bz_\ell)) = \occ(\Sp_{2g}(\bz / \ell \ZZ))
		\).
		Because $\Sp_{2g}(\bz / \ell \ZZ)$ is not simple, every element of $\occ(\Sp_{2g}(\bz / \ell \ZZ))$ is bounded in size by $|\Sp_{2g}(\bz / \ell \ZZ)| / 2$, so every element of $\quo(H_{\{\ell \le p\}})$ is bounded in size by $|\Sp_{2g}(\bz / p \ZZ)| / 2$. Observing that
		\[
			\frac{1}{2} \cdot |\Sp_{2g}(\bz / p \bz)| < |\PSp_{2g}(\bz / \ell \bz)|
		\]
		for every $\ell > p$, the desired condition follows by applying Lemma~\ref{lemma:simple-quotients-of-symplectic-group}.
	\end{proof}

	
	\subsection{Open Subgroups of $\GSp_{2g}(\wh{\ZZ})$}
	\label{subsection:open-subgroups}
We now return to studying the general symplectic group $\GSp_{2g}(\wh{\ZZ})$. The main result of this subsection tells us that the closure of the commutator subgroup of an open subgroup of $\GSp_{2g}(\wh{\ZZ})$ is open:


\begin{proposition} \label{theorem:commutator-open}
Let $g \geq 2$, and let $H \subset \GSp_{2g}(\zh)$ be an open subgroup. Then the closure of $[H, H]$ is an open subgroup of $\Sp_{2g}(\zh)$.
\end{proposition}

In order to prove Proposition~\ref{theorem:commutator-open}, we shall require a number of preliminary lemmas, which are stated and proven in Sections~\ref{sec1} and~\ref{sec2}.

\subsubsection{Openness Condition}\label{sec1}

The next two lemmas give us a criterion for openness in $\Sp_{2g}(\zh)$:

\begin{lemma}
		\label{lemma:open-image-in-finite-set}
		Let $S$ be a finite set of prime numbers, and let $H \subset \prod_{\ell \in S} \Sp_{2g}(\bz_\ell)$ be a closed subgroup. If each $H_\ell \subset \Sp_{2g}(\bz_\ell)$ is open, then $H \subset \prod_{\ell \in S} \Sp_{2g}(\bz_\ell)$ is open.
	\end{lemma}
	\begin{proof}
		There exists a finite-index subgroup $H' \subset H$ such that $H'(\ell)$ is trivial for every $\ell \in S$, namely the intersection of the kernels of the mod-$\ell$ reductions maps $H \to H(\ell)$. Since each $H'_\ell$ is a pro-$\ell$ group, Theorem~\ref{theorem:goursat} implies that $H' = \prod_{\ell \in S} H'_\ell$.
		Thus, $H$ contains an open subgroup and is therefore itself open.
	\end{proof}

       \begin{lemma}
		\label{theorem:adelic-open}
		Let $g \geq 2$ and let $H \subset \Sp_{2g}(\zh)$ be a closed subgroup. If $H_{\ell'}$ is open in $\Sp_{2g}(\bz_{\ell'})$ for all $\ell'$ and $H_\ell = \Sp_{2g}(\bz_\ell)$ for all but finitely many $\ell$, then $H$ is open in $\Sp_{2g}(\zh)$.
	\end{lemma}
	\begin{proof}
		Let $p$ be the largest prime with $H_{p} \neq \Sp_{2g}(\mathbb Z_{p})$. By Lemma~\ref{lemma:open-image-in-finite-set}, we have that  $H_{\{\ell \le p\}} \subset \prod_{\ell \le p} \Sp_{2g}(\mathbb Z_\ell)$ is an open subgroup.
		The result then follows from Proposition~\ref{theorem:truncate}.
	\end{proof}

\subsubsection{Two Computational Lemmas}\label{sec2}

The next two results are used in the proof of Proposition~\ref{theorem:commutator-open}. The following lemma describes the commutator of an element of $\Gamma_{\ell^m}$ with an element of $\Gamma_{\ell^n}$.

	\begin{lemma} \label{lemma:commutator-formula}
		Let $n\le m$ be positive integers, and let $\id_{2g} + \ell^n U$ and $\id_{2g} + \ell^m V$ be elements of $\GL_{2g}(\bz_\ell)$. Then we have
		\begin{align*}
		& (\id_{2g} + \ell^nU)^{-1}(\id_{2g} + \ell^mV)(\id_{2g} + \ell^n U)(\id_{2g} + \ell^m V)^{-1} \equiv \id_{2g} + \ell^{n+m} (VU - UV)\,\,\, (\on{mod}{\ell^{2n+m}}).
		\end{align*}
	\end{lemma}
	\begin{proof}
		We have
		\begin{align*}
		(\id_{2g} + \ell^mV)(\id_{2g} + \ell^n U)(\id_{2g} + \ell^mV)^{-1} &= \id_{2g} + \ell^n (\id_{2g} + \ell^mV)U(\id_{2g} + \ell^mV)^{-1} \\			
		&= \id_{2g} + \ell^n (\id_{2g} + \ell^m V)U \left(\sum_{i=0}^\infty (-1)^i \ell^{im} V^i\right) \\
		&= \id_{2g} + \ell^n \sum_{i=0}^\infty \Big[ (-1)^i \ell^{im} U V^i + (-1)^i \ell^{(i+1)m} VUV^i \Big]\\
		&= \id_{2g} + \ell^n U + \ell^{n+m} (VU - UV) (\id_{2g} + \ell^m V)^{-1}.
		\end{align*}
		Multiplying on the left by $(\id_{2g} + \ell^nU)^{-1}$ gives the desired result.
	\end{proof}

In the next proposition, we show the commutator subalgebra of $\mf{sp}_{2g}(\ZZ/\ell \ZZ)$ is sufficiently large for all primes $\ell$.

    \begin{proposition}\label{stopdrop}
We have the following results:
\begin{enumerate}
\item For all $g \geq 1$ and $\ell \geq 3$ we have $[\mf{sp}_{2g}(\ZZ/ \ell \ZZ), \mf{sp}_{2g}(\ZZ/ \ell \ZZ)] = \mf{sp}_{2g}(\ZZ/ \ell \ZZ)$.
\item For all $g \geq 1$ we have $[\mf{sp}_{2g}(\ZZ/ 4 \ZZ), \mf{sp}_{2g}(\ZZ/ 4 \ZZ)] \supset 2 \cdot \mf{sp}_{2g}(\ZZ / 2 \ZZ)$.
\end{enumerate}
\end{proposition}
\begin{proof}
	Statement (a) follows immediately from~\cite[Theorem 2.6]{eliotsteinclub}, which states that $\mf{sp}_{2g}(\ZZ/ \ell \ZZ)$ is simple for $\ell \geq 3$. It remains to prove Statement (b). For this, we compute several commutators and make deductions based on each one. For convenience, let $\mf{g} = [\mf{sp}_{2g}(\ZZ/ 4 \ZZ), \mf{sp}_{2g}(\ZZ/ 4 \ZZ)]$, let $A, D$ denote arbitrary $g \times g$ matrices, and let $B,C,E,F$ denote symmetric $g \times g$ matrices. \mbox{Since}
\begin{align}
	\label{equation:block-diagonal-commutator}
 \left[ \left[\begin{array}{c|c} A & 0 \\ \hline 0 & -A^T \end{array}\right], \left[\begin{array}{c|c} D & 0 \\ \hline 0 & -D^T \end{array}\right] \right] & = \left[\begin{array}{c|c} AD - DA & 0 \\ \hline 0 & A^TD^T - D^TA^T \end{array}\right],
 \intertext{all block-diagonal matrices in $\mf{sp}_{2g}(\ZZ/4\ZZ)$ with every diagonal entry equal to $0$ are contained in $\mf{g}$. This can be seen
by taking $A$ and $D$ to be various elementary matrices.
Furthermore,}
\label{equation:block-off-diagonal-commutator}
 \left[ \left[\begin{array}{c|c} 0 & B \\ \hline C & 0 \end{array}\right], \left[\begin{array}{c|c} 0 & E \\ \hline F & 0 \end{array}\right] \right] & = \left[\begin{array}{c|c} BF - EC & 0 \\ \hline 0 & CE - FB \end{array}\right],
 \intertext{so we can arrange that $BF-EC$ is an elementary matrix with a single nonzero entry on the diagonal.
	 Summing matrices from~\eqref{equation:block-diagonal-commutator} and~\eqref{equation:block-off-diagonal-commutator} tells us that all block-diagonal matrices are contained in $\mf{g}$. Additionally,}
\label{equation:identity-commutator}
  \left[ \left[\begin{array}{c|c} \id_g & 0 \\ \hline 0 & -\id_g \end{array}\right], \left[\begin{array}{c|c} 0 & B \\ \hline 0 & 0 \end{array}\right] \right] & = \left[\begin{array}{c|c} 0 & 2B \\ \hline 0 & 0 \end{array}\right].
  \intertext{Repeating the computation from~\eqref{equation:identity-commutator} with the other off-diagonal block nonzero implies that $2$ times any matrix in $\mf{sp}_{2g}(\ZZ / 2 \ZZ)$ whose diagonal blocks are $0$ is an element of $\mf{g}$. The desired result follows because $2 \cdot \mf{sp}_{2g}(\mathbb Z/2\ZZ)$ is contained in the subspace generated by the matrices from~\eqref{equation:block-diagonal-commutator},~\eqref{equation:block-off-diagonal-commutator}, and~\eqref{equation:identity-commutator}. \nonumber \qedhere}
\end{align}
\end{proof}

\subsubsection{Completing the Proof}

In order to prove Proposition~\ref{theorem:commutator-open}, we require the following lemma, which states that the closure of the commutator $[\Gamma_{\ell^k}, \Gamma_{\ell^k}]$ is large.

\begin{lemma} \label{proposition:commutator-effective}
	Fix $k \geq 1$. Then if $\ell \neq 2$, the closure of $[\Gamma_{\ell^k}, \Gamma_{\ell^k}]$ contains $\Gamma_{\ell^{2k}}$ and if $\ell = 2$, the closure of $[\Gamma_{\ell^k}, \Gamma_{\ell^k}]$ contains $\Gamma_{\ell^{2k+1}}$.
\end{lemma}
\begin{proof}
First suppose $\ell \geq 3$. Statement (1) of Proposition~\ref{stopdrop} implies that
for any $W' \in \mf{sp}_{2g}(\ZZ/ \ell \ZZ)$, there exist $U', V' \in \mf{sp}_{2g}(\ZZ/ \ell \ZZ)$ so that
$V'U' - U'V' = W'$.
Choosing lifts $W, U, V$ of $W', U', V'$, it follows from Lemma~\ref{lemma:commutator-formula} that for every $i$ and for every such
\begin{align*}
\id_{2g} + \ell^{2k+i} W \in \Gamma_{\ell^{2k+i}}, \quad
\id_{2g} + \ell^k U \in \Gamma_{\ell^k}, \quad \text{ and} \quad
\id_{2g} + \ell^{k+i} V \in \Gamma_{\ell^{k+i}},
\end{align*}
we have that
	\begin{align*}
		& (\id_{2g} + \ell^k U)^{-1}(\id_{2g} + \ell^{k+i} V)(\id_{2g} + \ell^k U)(\id_{2g} + \ell^{k+i} V)^{-1} \equiv \id_{2g} + \ell^{2k+i} W\,\,\, (\on{mod}{\ell^{2k+i+1}}).
		\end{align*}
Take $M_0 \in \Gamma_{\ell^{2k}}$. There exists $X_1 \in [\Gamma_{\ell^{2k}}, \Gamma_{\ell^{2k}}]$ and $M_1 \in \Gamma_{\ell^{2k+1}}$ with the property that $M_0 = X_1M_1$. Proceeding inductively in this manner, we obtain sequences $\{X_i : i = 1, 2, \dots\} \subset [\Gamma_{\ell^k}, \Gamma_{\ell^k}]$ and $\{M_i : i = 0, 1, 2, \dots\}$ with $M_i \in \Gamma_{\ell^{2k+i}}$ such that $M_i = X_{i+1}M_{i+1}$ for each $i$. Then we have that
$$M_0 = \lim_{i \to \infty} \left(\prod_{j = 1}^{i} X_j \right)M_i = \prod_{j = 1}^\infty X_j.$$
It follows that $\Gamma_{\ell^{2k}}$ is contained in the closure of $[\Gamma_{\ell^k}, \Gamma_{\ell^k}]$.

Now suppose $\ell = 2$. Observe that for each $k \geq 2$ we have
$$\id_{2g} + 2^k \cdot \mf{sp}_{2g}(\ZZ/ 4 \ZZ) = \ker(\Sp_{2g}(\ZZ/ 2^{k+2} \ZZ) \to \Sp_{2g}(\ZZ/ 2^k \ZZ)).$$
It follows from Statement (2) of Proposition~\ref{stopdrop} and Lemma~\ref{lemma:commutator-formula} that for every choice of $\id_{2g} + 2^{2k+i+1} W \in \Gamma_{2^{2k+i+1}}$ and for each nonnegative integer $i$, there exist $\id_{2g} + 2^k U \in \Gamma_{2^k}$ and $\id_{2g} + 2^{k+i} V \in \Gamma_{2^{k+i}}$ with the property that
	\begin{align*}
		& (\id_{2g} + 2^k U)^{-1}(\id_{2g} + 2^{k+i} V)(\id_{2g} + 2^k U)(\id_{2g} + 2^{k+i} V)^{-1} \equiv \id_{2g} + 2^{2k+i+1} W\,\,\, (\on{mod}{\ell^{2k+i+2}}).
		\end{align*}
One may now finish the proof by applying a similar inductive argument to the one used in the case $\ell \geq 3$.
\end{proof}

We are finally in position to prove the main result of this section.
\begin{proof}[Proof of Proposition~\ref{theorem:commutator-open}]
By Lemma~\ref{theorem:adelic-open}, it suffices to prove the following two statements:
\begin{enumerate}
\item The closure of $[H,H]$ surjects onto $\Sp_{2g}(\bz_\ell)$ for all but finitely many $\ell$.
\item The closure of $[H,H]$ maps onto an open subgroup of $\Sp_{2g}(\bz_\ell)$ for each $\ell$.
\end{enumerate}
For Statement (a), notice that $H$ surjects onto $\GSp_{2g}(\bz_\ell)$ for all but finitely many $\ell$.  Note that for $\ell \geq 3$, we have $[\GSp_{2g}(\bz_\ell), \GSp_{2g}(\bz_\ell)] = \Sp_{2g}(\bz_\ell)$ because, by~\cite[Proposition 3]{landesman-swaminathan-tao-xu:lifting-symplectic-group}, we have that
$$\Sp_{2g}(\ZZ_\ell) = [\Sp_{2g}(\ZZ_\ell), \Sp_{2g}(\ZZ_\ell)] \subset [\GSp_{2g}(\bz_\ell), \GSp_{2g}(\bz_\ell)] \subset \Sp_{2g}(\bz_\ell).$$
Thus, $[H,H]$ itself surjects onto $[\GSp_{2g}(\bz_\ell), \GSp_{2g}(\bz_\ell)] = \Sp_{2g}(\bz_\ell)$ for all $\ell \geq 3$.
		
To show statement (b), we prove that the closure of $[H', H']$ is open in $\Sp_{2g}(\bz_\ell)$ for any open subgroup $H' \subset \GSp_{2g}(\bz_\ell)$. Since $H'$ is open, there exists some $k \geq 1$ such that $\Gamma_{\ell^k} \subset H'$, so by Lemma~\ref{proposition:commutator-effective}, there exists $m \geq 2k$ such that $\Gamma_{\ell^m} \subset [\Gamma_{\ell^k}, \Gamma_{\ell^k}] \subset [H', H']$. Thus, $[H', H']$ contains an open subgroup and must therefore itself be open, as desired.
\end{proof}
