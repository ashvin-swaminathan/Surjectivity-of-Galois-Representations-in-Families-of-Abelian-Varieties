\section{Applications of Theorem~\ref{theorem:main}}\label{iknewyouweretrouble}

The purpose of this section is to demonstrate that the main result, Theorem~\ref{theorem:main}, can be applied to a number of interesting families of PPAVs, such as families containing a dense open substack of the locus of Jacobians of hyperelliptic curves, trigonal curves, or plane curves. In Section~\ref{crit}, we prove a general tool that is needed to guarantee big monodromy for the loci in our applications, and in Section~\ref{androidbeatsios}, we examine each of these \mbox{applications in detail.}

\subsection{Finite-Index Criterion}\label{crit}

In this section we prove Proposition~\ref{proposition:stacky-dominant-map-surjective-fundamental-group}, which will be applied
in the setting of Theorem~\ref{theorem:main} to determine
that $U$ has big monodromy when its image in the moduli stack of
abelian varieties has big monodromy. We begin by recalling an elementary criterion giving surjectivity for the map on \'{e}tale fundamental groups induced by a morphism of algebraic stacks.
\begin{lemma}
       \label{lemma:surjectivity-criterion-pi1}
       Suppose $f\colon X \rightarrow Y$ is a map of algebraic stacks.
       The fiber product $U \times_Y X$ is connected
       for all finite connected \'etale maps $U \rightarrow Y$,
       if and only if
       the induced map $\pi_1(X) \rightarrow \pi_1(Y)$ is surjective.
       In particular, if $X$ and $Y$ are normal, integral, and Noetherian, and $f: X \rightarrow Y$ is a flat map with connected geometric
       generic fiber, then the induced map $\pi_1(X) \rightarrow \pi_1(Y)$ is surjective.
\end{lemma}
\begin{proof}
Using the identification between connected finite \'etale covers of $Y$ and transitive $\pi_1(Y)$-sets, the first part is immediate from the fact that a continuous map $\phi \colon H \rightarrow G$ between profinite groups is surjective if and only if every finite discrete $G$-set with transitive $G$-action is transitive as a $H$-set.
The latter fact holds because the image of $H$ is closed, and so $\phi$ is not surjective if and only if there is some open normal
$N \subset G$ with $f(H) \subset N$, in which case $H$ does not acts transitively on $G/N$.

For the second statement,
we only need verify that a connected finite \'etale cover $U \rightarrow Y$ pulls back to a connected cover of $X$.
Note that because $X$ and $Y$ are normal and integral, \'etale covers of $X$ and $Y$ are connected if and only if they are irreducible. (Here, we are using that
normal and connected implies irreducible and that
normality is local in the \'etale topology over Noetherian stacks by Serre's R1 + S2 criterion for normality.)
Thus, we only need show that if $U \rightarrow Y$ is any irreducible finite \'etale cover, then so is $X \times_Y U \rightarrow X$.
But this follows from the assumptions that $f$ is flat and $U$ is integral,
which implies all generic points of $X \times_Y U$ map to the generic point of $U$.
So, if $X \times_Y U$ were reducible, the geometric generic fiber over $U$ would also be
reducible, which contradicts the assumption that $f$ has connected geometric generic fiber, since a geometric generic fiber of
$X \times_Y U$ is also a geometric generic fiber of $f$.
\end{proof}

\begin{proposition} \label{proposition:stacky-dominant-map-surjective-fundamental-group}
Let $k$ be an arbitrary field of characteristic $0$. Suppose $X$ is a scheme and $Y$ is a Deligne-Mumford stack over $k$, both of which are normal, integral, separated, and finite type over $k$, and let $f\colon X \rightarrow Y$ be a dominant map.
Then, the image of the induced map
       $\pi_1(X) \rightarrow \pi_1(Y)$ has finite index in $\pi_1(Y)$. If, in addition, the geometric
       generic fiber of $f$ is connected, then the map $\pi_1(X) \rightarrow \pi_1(Y)$ is surjective.
\end{proposition}
\begin{proof}
To begin, we reduce to the case in which $f$ is smooth.
By generic smoothness, we may replace $X$
by a dense open $X' \subset X$ so that $f|_{X'}$ is smooth.
Since, $\pi_1(X') \rightarrow \pi_1(X)$ is a surjection by Lemma~\ref{lemma:surjectivity-criterion-pi1}, in order to prove the proposition, we may replace $X$ by $X'$.

The last sentence of this Proposition follows from Lemma~\ref{lemma:surjectivity-criterion-pi1} (here we only needed that the map be $f$ be flat, but we have already reduced to the case it is smooth). To conclude, we only need prove that the image of $\pi_1(X)\rightarrow \pi_1(Y)$ has finite index in $\pi_1(Y)$, without the assumption that the geometric generic fiber of $f$ is connected. Since $f$ is smooth and $Y$ is Deligne-Mumford, we can find a scheme $U$ and a dominant \'etale map $U \rightarrow X$ so that $U \rightarrow Y$ factors through $\mathbb A^N_Y$ where $N$ is the dimension of the geometric generic fiber of $f$ and $U \rightarrow \mathbb A^N_Y$ \'etale.
So, after passing to a dense open substack of $W \subset \mathbb A^N_Y$ and a dense open subscheme $U' \subset U$, we may assume that $U' \rightarrow W$ is a finite \'etale cover:
To see why, take a smooth cover of $\mathbb A^N_Y$ by a scheme. The pullback to $U$ is a separated algebraic space, so it has a dense open subspace that is a scheme. The finiteness claim then follows because the resulting \'etale morphism of schemes is locally quasi-finite, of finite type, and quasi-separated, hence generically finite on the target.
Since $U \rightarrow W$ is finite \'etale, $\pi_1(U') \rightarrow \pi_1(W)$ has finite index.
Because the maps $\pi_1(W) \rightarrow \pi_1(\mathbb A^N_Y)$ and $\pi_1(\mathbb A^N_Y) \rightarrow \pi_1(Y)$ are surjective by Lemma~\ref{lemma:surjectivity-criterion-pi1}, the composition $\pi_1(U') \rightarrow \pi_1(Y)$ has finite index in $\pi_1(Y)$, and hence so does $\pi_1(X) \rightarrow \pi_1(Y)$.
\end{proof}

\subsection{Applications}\label{androidbeatsios}
Let $K$ be a number field with fixed algebraic closure $\ol{K}$, let $\mg$ denote the moduli stack of curves of genus $g$ over $K$,
and let $\ag$ denote the moduli stack of PPAVs of dimension $g$ over $K$. We have a natural map $\tau_g \colon \mg \rightarrow \ag$ given by the Torelli map, which sends a curve to its Jacobian. Let $\ug$ denote the universal family over $\ag$. Note that if $U$ is any scheme and $A \to U$ is a family of PPAVs, then there exist maps $A \to \ug$ and $U \to \ag$ so that $A$ equals the fiber product $U \times_\ag \ug$.

We will also be interested in the locus of smooth hyperelliptic curves of genus $g$, $\hyperelliptic g \subset \mg$, and locus of trigonal curves of genus $g$, $\trigonal g \subset \mg$.
If a curve $C$ is trigonal, there exists a unique nonnegative integer $M$, called the Maroni invariant, with the property that there is a canonical embedding into the Hirzebruch surface $\mathbb F_M \defeq \mathbb P_{\mathbb P^1} (\mathcal O_{\mathbb P^1} \oplus \mathcal O_{\mathbb P^1}(M))$. As mentioned in \cite{patel2015chow}, the Maroni invariant takes on all integer values between $0$ and $\frac{g+2}{3}$ with the same parity as $g$. Let $\trigonal g(M) \subset \mg$ denote the substack of trigonal \mbox{curves of Maroni invariant $M$.}

In order to more easily utilize Proposition~\ref{proposition:stacky-dominant-map-surjective-fundamental-group} for the purpose of giving interesting examples of Theorem ~\ref{theorem:main}, we record the following easy consequence of Proposition~\ref{proposition:stacky-dominant-map-surjective-fundamental-group}:
\vspace*{-0.2cm}
\begin{corollary} \label{corollary:criterion-for-applying-main}
       Let $U \subset \mathbb P^r_K$ be an open subscheme, and let $A \rightarrow U$ be a family of $g$-dimensional PPAVs. Let $\phi \colon U \rightarrow \ag$ be the map induced by the universal property of $\ag$.
Let $V$ be the smallest locally closed substack of $\ag$ through which $U$ factors, and let $W \subset \ag$ be a normal integral substack.
Suppose further that $W \cap V$ is dense in $W$ and that $V$ is normal.
Then, if $W$ has big monodromy, so do $V$ and $U$.
Furthermore, if the geometric generic fiber of $\phi$ is irreducible, then the monodromy of $V$ agrees with that of $U$. In particular, the conclusion of Theorem ~\ref{theorem:main}
holds for $U$.
\end{corollary}
\vspace*{-0.4cm}
\begin{proof}
	By Lemma~\ref{lemma:surjectivity-criterion-pi1}, if $W$ has big monodromy so does the dense open subset $W \cap V \subset W$.
	Therefore, $V$ has big monodromy, because it contains $W \cap V$, which has big monodromy.
	The result then follows from Proposition~\ref{proposition:stacky-dominant-map-surjective-fundamental-group},
once we verify that both $U$ and $V$ are normal, irreducible, separated, and finite type over $K$, with $V$ Deligne-Mumford.
All of these conditions are immediate except possibly that $V$
is generically smooth, which holds by generic smoothness on a smooth \mbox{cover of $V$ by a scheme.}
\end{proof}
\vspace*{-0.2cm}
We are now in position to state and prove the main theorem of this section:
\vspace*{-0.2cm}
\begin{theorem} \label{corollary:examples}
Suppose $A \rightarrow U$ is a rational family of principally polarized abelian varieties
and define $V$ to be the smallest locally closed substack of $\ag$ through which $U$ factors.
The conclusion of Theorem~\ref{theorem:main} holds whenever $V$ is normal and contains a dense open substack of one of the following loci:
\begin{enumerate}
	       \item[\customlabel{big-hyperelliptic}{(a)}] The locus $\tau_g(\hyperelliptic g)$ for any $g \geq 0$.
		      For every $g \geq 0$ exists a $U$ dominating $\tau_g(\hyperelliptic g)$ because $\hyperelliptic g$ is unirational.
               \item[\customlabel{big-maroni}{(b)}] The locus $\tau_g(\trigonal g(M))$ of Jacobians of trigonal curves with Maroni invariant $M < \frac{g}{3}-1$ for any $g \geq 5$. In this case, there exists $U$ dominating $\tau_g(\trigonal g(M))$ because \mbox{$\trigonal g(M)$ is unirational.}
		\item[\customlabel{big-trigonal}{(c)}] The locus of trigonal curves $\trigonal g$ in any $g \geq 3$.
We can take $U$ to be any open subscheme of $\trigonal g$, as $\trigonal g$ is rational.
               \item[\customlabel{big-plane}{(d)}] The locus of Jacobians of degree-$d$ smooth plane curves for any $d \geq 3$. In this case, the open subscheme of the Hilbert scheme
of degree-$d$ plane curves parameterizing smooth curves is rational and dominates the locus of Jacobians of degree-$d$ plane curves.
\item[\customlabel{big-mg}{(e)}] The locus $\tau_g(\mg)$ for any $g \geq 1$. In this case, when $1 \leq g \leq 14$, $\mg$ is unirational, so there exists a $U$ dominating $\mg$. Moreover, when $3 \leq g \leq 6$, $\mg$ is rational, and so we may take $U$ to be any open subscheme of $\mg$.
               \item[\customlabel{big-ag}{(f)}] The locus $\ag$ for any $g \geq 1$. When $1 \leq g \leq 5$, $\ag$ is unirational, so such a $U$ exists.
\end{enumerate}
\end{theorem}
\begin{proof}
By Corollary~\ref{corollary:criterion-for-applying-main}, it suffices to check that each of the families enumerated above has a dense open substack which has big monodromy, is irreducible, and is normal, and to verify the rationality and unirationality claims made above.
Irreducibility of these loci is well-known.
Note that in the first five of the following cases, if we denote the locus in question by $\tau_g(W) \subset \ag$, it suffices to verify that $W \subset \mg$ is smooth as a substack of $\mg$, as we now explain.
First, $\tau_g(W) \subset \ag$ is generically smooth because it is reduced, since it is the image of $W$, which is reduced.
Taking a smooth dense open $Z' \subset \tau_g(W)$, we have that $\tau_g^{-1}(Z') \subset W$ is a dense open substack, hence it is also
smooth and has big monodromy. This implies $Z'$ also has big monodromy since the monodromy of a locus in $\mg$ agrees with the monodromy of its image in $\ag$ under $\tau_g$, as both can be identified with the monodromy action on
the first cohomology group.
We now conclude the proof by verifying that each locus has a certain dense open substack which has big monodromy in $\mg$ (in the first five cases), is normal, and is rational or unirational when claimed.
\begin{enumerate}[(a)]
              \item The hyperelliptic locus, $\hyperelliptic g$, has big geometric monodromy as was shown independently in~\cite[Lemma 8.12]{mumford:tata-lectures-on-theta-ii} and~\cite[Th\'eor\`eme 1]{acampo:tresses-monodromie-et-le-groupe-symplectique}.
	       The hyperelliptic locus $\hyperelliptic g$ is smooth and unirational because it is the quotient of an open subscheme of
	       $\mathbb P^{2g+2}_K$ by the smooth action of $\PGL_2$.
\item By~\cite[Theorem, p.~2]{bolognesi2016mapping}, $\trigonal g (M)$ has big geometric monodromy when $M < \frac{g}{3}-1$.
	      Additionally, $\trigonal g(M)$ is smooth and unirational because it can be expressed as a quotient
	      $[U/G]$ of a smooth rational scheme $U$ by a smooth group scheme $G$.
	      Here, $G$ is the group of automorphisms of the Hirzebruch surface $\mathbb F_M$
	      and $U$ is an open subscheme of the projectivization of the linear system of class $3e + \left(\frac{g + 3M + 2}{2}\right) f$ on $\mathbb F_M$, where $f$ is the class of the fiber over $\mathbb P^1$ and $e$
	      is the unique section with negative self-intersection (see ~\cite[p.~8]{bolognesi2016mapping} for an explanation
		of this description of $U$).
	       Note that in this application, we are implicitly translating between
the topological monodromy representation of $\mg$ described in ~\cite[Theorem, p.~2]{bolognesi2016mapping} and the algebraic Galois representation in
$\ag$, but these two representations are compatible, essentially because
both are given by the action of the fundamental group on the first cohomology
group.
\item In the case that $g \geq 5$, we have $\trigonal g(g \bmod 2)$ is birational to $\trigonal g$, so $\trigonal g$
	has a smooth dense open with big geometric monodromy by the previous part.
	Next, $\trigonal g$ is rational for $g \geq 5$ by~\cite[Theorem, p.~1]{ma2014rationality}.
	The cases $g = 3, 4$ hold because for such $g$, $\trigonal g$ forms a dense open in $\mg$, which is itself rational and smooth,
	as shown in the proof of part~\ref{big-mg} below.
\item Take a smooth open substack $W$ of the (reduced)
substack of plane curves in $\mg$.
There is a dominant map from a dense open subscheme of the the Hilbert scheme of
degree-$d$ plane curves to $W$.
By~\cite[Th\'{e}or\`{e}me 4]{beauville1986groupe}, the locus of smooth degree-$d$ plane curves in the Hilbert scheme has big geometric monodromy. 
It follows from Lemma~\ref{lemma:surjectivity-criterion-pi1}
that $W$ has big monodromy.
The locus of \emph{smooth} degree-$d$ plane curves in the Hilbert scheme
is certainly rational,
       as it is an open subscheme of the Hilbert scheme of degree-$d$ plane curves, which is itself isomorphic to $\mathbb P_K^{\binom{d+2}{2}-1}$.
\item By~\cite[5.12]{deligne1969irreducibility}, the geometric monodromy of $\mg$ is all of $\Sp_{2g}(\widehat {\mathbb Z})$ for every $g \geq 1$. 	Next, $\mg$, smooth by \cite[Theorem 5.2]{deligne1969irreducibility}. We have that $\mg$ is unirational for $1 \leq g \leq 14$ by~\cite{verra2005unirationality}. Moreover, when $3 \leq g \leq 6$, we have that $\mg$ is rational; see~\cite[p.~2]{casnati2007rationality} for comprehensive references.
\item Note that $\ag$ has geometric big monodromy because $\ag$ contains $\mg$ and $\mg$ has monodromy $\Sp_{2g}(\widehat {\mathbb Z})$, as argued in point (d).
Further,
		       $\ag$ is smooth by \cite[Theorem 2.4.1]{oort:finite-group-schemes-local-moduli-for-abelian-varieties-and-lifting-problems}.
	We have that $\ag$ is unirational for $1 \leq g \leq 5$ \mbox{as shown in~\cite[p.~1]{verra2005unirationality}.}\qedhere
\end{enumerate}
\end{proof}

\begin{remark}
       \label{remark:mmmm}
       In most of the cases enumerated in Theorem~\ref{corollary:examples}, we actually know that the geometric monodromy is not only big, but also equal to $\Sp_{2g}(\widehat{\mathbb Z})$.
       By Corollary~\ref{corollary:criterion-for-applying-main},
       this occurs when $U$ has irreducible geometric generic fiber
       over any of the following loci:
       \begin{enumerate}
               \item The locus $\trigonal g (M)$ for any $M < \frac{g}{3}-1$, by~\cite[Theorem, p.~2]{bolognesi2016mapping};
               \item The locus of plane curves of degree $d$ with $d$ even, by~\cite[Theoreme 4(i)]{beauville1986groupe};
               \item The locus $\mg$ for any $g$, by~\cite[5.12]{deligne1969irreducibility};
	       \item The locus $\ag$ for any $g$, because $\mg \subset \ag$ and $\mg$ has full monodromy by point (d).
       \end{enumerate}
\end{remark}

\begin{remark}
       If $A \to U$ is a family with $\mono_A^{\on{geom}} = \Sp_{2g}(\zh)$, then the group $\mono_A$ can be determined as follows. The intersection $K \cap \bq^{\on{cyc}}$ is of the form $\bq(\zeta_n)$ for some $n \ge 2$. Let $r_n : \zh \to \bz / n \bz$ be the reduction map. Then
    \[
       \mono_A = \ker (r_n \circ \on{mult}) = \{M \in \Sp_{2g}(\zh) : \on{mult} M \equiv 1 \pmod{n} \},
    \]
       which follows from Remark~\ref{remark:det-rho-is-chi}.
Thus, when the conclusion of the preceding remark holds, Theorem~\ref{theorem:main} tells us the following:
    \begin{itemize}
           \item If $K\neq \bq$, or if $K = \bq$ and $g \ge 3$, then most $u \in U(K)$ have $\mono_{A_u} = \ker (r_n \circ \on{mult})$.
	   \item If $K = \bq$ and $g \in \left\{ 1,2 \right\}$, then most $u \in U(K)$ are such that $[\GSp_{2g}(\zh) : \mono_{A_u}] = 2$.
       \end{itemize}
\end{remark}

\begin{remark} \label{remark:}
	Theorem~\ref{corollary:examples}~\ref{big-hyperelliptic} tells us that if $U$ dominates $\hyperelliptic g$, then the conclusion of Theorem~\ref{theorem:main} holds for $U$. In the case where $U$ has irreducible geometric generic fiber, we can say explicitly what the monodromy group of the family is and what its commutator is. For example, let $\mathscr{Y}_{2g+2,K}$ denote the family of genus-$g$ hyperelliptic curves over $K$ with Weierstrass equation given by $y^2 = x^{2g+2} + a_{2g+1}x^{2g+1} + \dots + a_0$. We show in~\cite[Theorem 1.2]{landesman-swaminathan-tao-xu:hyperelliptic-curves} that most members of $\mathscr{Y}_{2g+2, K}$ have monodromy equal to $\mono_{\mathscr Y_{2g+2,K}}$ (which we explicitly compute) over $K \neq \QQ$, and have index-$2$ monodromy when $K = \QQ$. We neither prove nor state this result precisely here, but a complete statement and proof is given in~\cite{landesman-swaminathan-tao-xu:hyperelliptic-curves}.
\end{remark}
